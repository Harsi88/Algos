%%%%%%%%%%%%%%%%%%%%%%%%%%%%%%%%%%%%%%%%%
% Structured General Purpose Assignment
% LaTeX Template
%
% This template has been downloaded from:
% http://www.latextemplates.com
%
% Original author:
% Ted Pavlic (http://www.tedpavlic.com)
%
% Note:
% The \lipsum[#] commands throughout this template generate dummy text
% to fill the template out. These commands should all be removed when 
% writing assignment content.
%
%%%%%%%%%%%%%%%%%%%%%%%%%%%%%%%%%%%%%%%%%

%----------------------------------------------------------------------------------------
%	PACKAGES AND OTHER DOCUMENT CONFIGURATIONS
%----------------------------------------------------------------------------------------

\documentclass{article}

\usepackage{fixltx2e}
\usepackage{fancyhdr} % Required for custom headers
\usepackage{lastpage} % Required to determine the last page for the footer
\usepackage{extramarks} % Required for headers and footers
\usepackage{graphicx} % Required to insert images
\usepackage{lipsum} % Used for inserting dummy 'Lorem ipsum' text into the template

% Margins
\topmargin=-0.45in
\evensidemargin=0in
\oddsidemargin=0in
\textwidth=6.5in
\textheight=9.0in
\headsep=0.25in 

\linespread{1.1} % Line spacing

% Set up the header and footer
\pagestyle{fancy}
\lhead{\hmwkAuthorName} % Top left header
\chead{\hmwkClass\ (\hmwkClassInstructor\ \hmwkClassTime): \hmwkTitle} % Top center header
\rhead{\firstxmark} % Top right header
\lfoot{\lastxmark} % Bottom left footer
\cfoot{} % Bottom center footer
\rfoot{Page\ \thepage\ of\ \pageref{LastPage}} % Bottom right footer
\renewcommand\headrulewidth{0.4pt} % Size of the header rule
\renewcommand\footrulewidth{0.4pt} % Size of the footer rule

\setlength\parindent{0pt} % Removes all indentation from paragraphs

%----------------------------------------------------------------------------------------
%	DOCUMENT STRUCTURE COMMANDS
%	Skip this unless you know what you're doing
%----------------------------------------------------------------------------------------

% Header and footer for when a page split occurs within a problem environment
\newcommand{\enterProblemHeader}[1]{
\nobreak\extramarks{#1}{#1 continued on next page\ldots}\nobreak
\nobreak\extramarks{#1 (continued)}{#1 continued on next page\ldots}\nobreak
}

% Header and footer for when a page split occurs between problem environments
\newcommand{\exitProblemHeader}[1]{
\nobreak\extramarks{#1 (continued)}{#1 continued on next page\ldots}\nobreak
\nobreak\extramarks{#1}{}\nobreak
}

\setcounter{secnumdepth}{0} % Removes default section numbers
\newcounter{homeworkProblemCounter} % Creates a counter to keep track of the number of problems

\newcommand{\homeworkProblemName}{}
\newenvironment{homeworkProblem}[1][Problem \arabic{homeworkProblemCounter}]{ % Makes a new environment called homeworkProblem which takes 1 argument (custom name) but the default is "Problem #"
\stepcounter{homeworkProblemCounter} % Increase counter for number of problems
\renewcommand{\homeworkProblemName}{#1} % Assign \homeworkProblemName the name of the problem
\section{\homeworkProblemName} % Make a section in the document with the custom problem count
\enterProblemHeader{\homeworkProblemName} % Header and footer within the environment
}{
\exitProblemHeader{\homeworkProblemName} % Header and footer after the environment
}

\newcommand{\problemAnswer}[1]{ % Defines the problem answer command with the content as the only argument
\noindent\framebox[\columnwidth][c]{\begin{minipage}{0.98\columnwidth}#1\end{minipage}} % Makes the box around the problem answer and puts the content inside
}

\newcommand{\homeworkSectionName}{}
\newenvironment{homeworkSection}[1]{ % New environment for sections within homework problems, takes 1 argument - the name of the section
\renewcommand{\homeworkSectionName}{#1} % Assign \homeworkSectionName to the name of the section from the environment argument
\subsection{\homeworkSectionName} % Make a subsection with the custom name of the subsection
\enterProblemHeader{\homeworkProblemName\ [\homeworkSectionName]} % Header and footer within the environment
}{
\enterProblemHeader{\homeworkProblemName} % Header and footer after the environment
}
   
%----------------------------------------------------------------------------------------
%	NAME AND CLASS SECTION
%----------------------------------------------------------------------------------------

\newcommand{\hmwkTitle}{Assignment\ \#1} % Assignment title
\newcommand{\hmwkDueDate}{Sunday,\ August\ 25,\ 2013} % Due date
\newcommand{\hmwkClass}{CSL\ 356} % Course/class
\newcommand{\hmwkClassTime}{} % Class/lecture time
\newcommand{\hmwkClassInstructor}{S. Iyengar} % Teacher/lecturer
\newcommand{\hmwkAuthorName}{Harsimran Singh} % Your name

%----------------------------------------------------------------------------------------
%	TITLE PAGE
%----------------------------------------------------------------------------------------

\title{
\vspace{2in}
\textmd{\textbf{\hmwkClass:\ \hmwkTitle}}\\
\normalsize\vspace{0.1in}\small{Due\ on\ \hmwkDueDate}\\
\vspace{0.1in}\large{\textit{\hmwkClassInstructor\ \hmwkClassTime}}
\vspace{3in}
}

\author{\textbf{\hmwkAuthorName}}
\date{} % Insert date here if you want it to appear below your name

%----------------------------------------------------------------------------------------

\begin{document}

\maketitle

%----------------------------------------------------------------------------------------
%	TABLE OF CONTENTS
%----------------------------------------------------------------------------------------

%\setcounter{tocdepth}{1} % Uncomment this line if you don't want subsections listed in the ToC

\newpage
\tableofcontents
\newpage

%----------------------------------------------------------------------------------------
%	PROBLEM 1
%----------------------------------------------------------------------------------------

% To have just one problem per page, simply put a \clearpage after each problem

\begin{homeworkProblem}
{\bf Give an algorithm to detect whether a given undirected graph contains a cycle.\
If the graph contains a cycle then your algorithm should output one.(It should \
not output all the cycles of the graph it should output just one of them). The \
running time of your algorithm should be \emph{O(m+n)} for a graph with n nodes \
and m edges.}
\vspace{10pt} % Question

\problemAnswer{ % Answer
\begin{center}
{\bf Solution }
\end{center}
Depth first search can be used to check whether an undirected graph contains a cycle\
or not. Initially all the node are marked not visited. Start from a node, say x\textsubscript{i} and mark it visited.
Then call the DFS on children of this node recursively and make x\textsubscript{i} \
(the node which called the DFS on them) their parent. While doing this if you encounter\
a visited node than you have found a cycle (because both nodes were already connected in DFS 
tree and now we have found an edge between them, hence a cycle). Now take these nodes and by backtracing their\
parent nodes we will get that cycle.

Because, we are performing DFS which takes \emph{O(m+n)} to complete, our algorithm's time 
complexity is also \emph{O(m+n)}.
}
\end{homeworkProblem}

%----------------------------------------------------------------------------------------
%	PROBLEM 2
%----------------------------------------------------------------------------------------

\begin{homeworkProblem}
{\bf Inspired by the example of that great Cornellian, Vladimir Nabokov, some 
of your friends have become amateur lepidopterists (they study butterflies). \
Often when they return from a trip with specimens of butterflies, 
it is very difficult for them to tell how many distinct species they've 
caught-thanks to the fact that many species look very similar to one 
another. One day they return with n butterflies, and they believe that each 
belongs to one of two different species, which we'll call A and B for 
purposes of this discussion. They'd like to divide the n specimens into 
two groups-those that belong to A and those that belong to B-but it's 
very hard for them to directly label anyone specimen. So they decide to 
adopt the following approach. 
For each pair of specimens i and j, they study them carefully side by 
side. If they're confident enough in their judgment, then they label the 
pair (i,j) either \lq\lq same\rq\rq\@ (meaning they believe them both to come from 
the same species) or \lq\lq different\rq\rq\@ (meaning they believe them to come from 
different species). They also have the option of rendering no judgment 
on a given pair, in which case we'll call the pair ambiguous. 
So now they have the collection of n specimens, as well as a collection 
of m judgements (either \lq\lq same\rq\rq\@ or \lq\lq different\rq\rq) for the pairs that were not 
declared to be ambiguous. They'd like to know if this data is consistent 
with the idea that each butterfly is from one of species A or B. So more 
concretely, we'll declare the m judgements to be consistent if it is possible 
to label each specimen either A or B in such a way that for each pair (i,j) 
labeled \lq\lq same\rq\rq\@, it is the case that i and j have the same label; and for each 
pair (i,j) labeled \lq\lq different\rq\rq\@, it is the case that i and j have different labels. 
They're in the middle of tediously working out whether their judgements 
are consistent, when one of them realizes that you probably have an 
algorithm that would answer this question right away. 
Give an algorithm with running time \emph{O(m + n)} that determines 
whether the m judgements are consistent. 
}
\vspace{10pt} % Question

\problemAnswer{ % Answer
\begin{center}
{\bf Solution }
\end{center}
Let G=(V,E) be a graph  with all the butterflies as their vertices and judgements as edges
between the vertices.
Initially all vertices are white. Start with DFS on a node, say x\textsubscript{r} and turn it grey. Now call DFS on children,
say x\textsubscript{i}, x\textsubscript{i+1}, ..., x\textsubscript{i+j} recursively. Check the label
of all the pairs generated at each call (a parent and a child) ( x\textsubscript{a}, x\textsubscript{b} ).
Let us say x\textsubscript{a} is grey in color.
Color x\textsubscript{b} grey ( same as x\textsubscript{a} ) if label is same, otherwise
color x\textsubscript{b} black (different from x\textsubscript{a} ). 

We say two nodes are consistent, if there is an edge between them and

label is ``same'' $\Rightarrow$ both are same in color and

label is ``different'' $\Rightarrow$ both are different in color 

\

Now, if during this algorithm we encounter a node x\textsubscript{n} which is already
colored. If node which called DFS on x\textsubscript{n}, say x\textsubscript{m} is 
consistent with x\textsubscript{n} then we carry on otherwise we say there is 
inconsistency in their judgements.

\

Because we are using DFS, therefore order of our algorithm is \emph{O(m+n)}.
}
\end{homeworkProblem}


%----------------------------------------------------------------------------------------
%	PROBLEM 3
%----------------------------------------------------------------------------------------

\begin{homeworkProblem}
{\bf We have a connected graph \emph{G = (V, E)} and a specific vertex U $\in$ V. Suppose 
we compute a depth-first search tree rooted at u, and obtain a tree T that 
includes all nodes of G. Suppose we then compute a breadth-first search 
tree rooted at u, and obtain the same tree T. Prove that G = T. (In other 
words, if T is both a depth-first search tree and a breadth-first search 
tree rooted at u, then G cannot contain any edges that do not belong to 
T.) 
}
\vspace{10pt} % Question

\problemAnswer{ % Answer
\begin{center}
{\bf Solution }
\end{center}
Let us assume that $\exists$ an edge (s,t) in \emph{G = (V,E)} which is not present 
in Tree T.

If s and t has an edge between them in the graph then either of s or t must be 
ancester of other in DFS tree T.

$\Rightarrow$ $\exists$ another path between s and t in graph that is included in tree T,
because edge (s,t) is not in the tree.
Let this path be s, x\textsubscript{1}, x\textsubscript{2},...,x\textsubscript{j}, t.
Let L\textsubscript{s} be the layer of s in tree T and similarly, L\textsubscript{t}
is the layer of t in tree T. Because $\exists$ at least one vertex in the path from s 
and t in DFS tree T $\Rightarrow$ difference between L\textsubscript{s} and L\textsubscript{t} 
is at least 2.

\

Now while building BFS tree (which should be same as DFS tree T), because s and t has edge 
between them, so they will either belong to same layer L\textsubscript{s} or two different
layers L\textsubscript{s} and L\textsubscript{t} such that difference between layers is 1.

But in DFS tree T, L\textsubscript{s} and L\textsubscript{t} are at least two layers apart.
$\Rightarrow$ DFS tree T is not same as BFS tree, but we are given that both trees are same.

Hence, Contradiction.

Therefore, $\exists$ no edge (s,t) which is present in G but not in Tree T $\Rightarrow$ G=T.
}
\end{homeworkProblem}
%----------------------------------------------------------------------------------------
%	PROBLEM 4
%----------------------------------------------------------------------------------------

\begin{homeworkProblem}
{\bf Some friends of yours work on wireless networks, and they're currently 
studying the properties of a network of n mobile devices. As the devices 
move around (actually, as their human owners move around), they define 
a graph at any point in time as follows: there is a node representing each 
of the n devices, and there is an edge between device i and device j if the 
physical locations of i and j are no more than 500 meters apart. (If so, we 
say that i and j are \lq\lq in range\rq\rq\@ of each other.)
They'd like it to be the case that the network of devices is connected at 
all times, and so they've constrained the motion of the devices to satisfy
the following property: at all times, each device i is within 500 meters 
of at least n/2 of the other devices. (We'll assume n is an even number.) 
What they'd like to know is: Does this property by itself guarantee that 
the network will remain connected? 
Here's a concrete way to formulate the question as a claim about 
graphs.

\

\emph{Claim: Let G be a graph on n nodes, where n is an even number. If every node 
of G has degree at least n/2, then G is connected. }

\

Decide whether you think the claim is true or false, and give a proof of 
either the claim or its negation.
}
\vspace{10pt} % Question

\problemAnswer{ % Answer
\begin{center}
{\bf Solution }
\end{center}
Let us assume that graph G = (V,E) with n nodes (n is even) is not connected $\Rightarrow$ 
$\exists$ two nodes s and t in the graph
such that there is no path between them.

\emph{Given:} s and t both have degree at least n/2.\

Let say s has degree i and has edge with vertices V\textsubscript{s} = \{ s\textsubscript{1}, s\textsubscript{2},...,s\textsubscript{i}\}
and t has degree j and has edge with vertices V\textsubscript{t} = \{ t\textsubscript{1}, t\textsubscript{2},...,t\textsubscript{j}\}.\
Both i and j are greater than n/2.\

If there is no path between them $\Rightarrow$
V\textsubscript{s} and V\textsubscript{t} are disjoint.
This will result in number of nodes $\geq$ n+2 

Number of nodes = ($\mid$\{ s \}$\mid$ + $\mid$\{ t \}$\mid$ + $\mid$\{ V\textsubscript{s} \}$\mid$ + $\mid$\{ V\textsubscript{t} \}$\mid$

Number of nodes $\geq$ 1+1+n/2+n/2

Number of nodes $\geq$ n+2

But our graph G has only n nodes.

Hence contradiction.

Thus, our assumption is false.

Therefore, $\exists$ no s and t such that there is no path from s to t.

Hence, graph G is connected.
}
\end{homeworkProblem}

%----------------------------------------------------------------------------------------
%	PROBLEM 5
%----------------------------------------------------------------------------------------

\begin{homeworkProblem}
{\bf There's a natural intuition that two nodes that are far apart in a 
communication network-separated by many hops-have a more tenuous 
connection than two nodes that are close together. There are a number 
of algorithmic results that are based to some extent on different ways of 
making this notion precise. Here's one that involves the susceptibility of 
paths to the deletion of nodes. 

Suppose that an n-node undirected graph G = (V, E) contains two 
nodes s and t such that the distance between s and t is strictly greater 
than n/2. Show that there must exist some node v, not equal to either s 
or t, such that deleting v from G destroys all s-t paths. (In other words, 
the graph obtained from G by deleting v contains no path from s to t.) 
Give an algorithm with running, time O(m + n) to find such a node v
}
\vspace{10pt} % Question

\problemAnswer{ % Answer
\begin{center}
{\bf Solution }
\end{center}
Distance between s and t is strictly greater than n/2 $\Rightarrow$ there are at least 
n/2 nodes between s and t, let say a\textsubscript{1}, a\textsubscript{2},..., a\textsubscript{m}
and m $\geq$ n/2.

If $\exists$ a vertex a\textsubscript{i} which is not included in any cycle formation in the graph than 
this vertex can be removed to disconnect s from t, because every path from s to t must go through 
this vertex a\textsubscript{i}.

Now, We will prove that such vertex always exists under given conditions.

Let say $\exists$ a path between two vertices a\textsubscript{i} and a\textsubscript{j} other 
than a\textsubscript{i}, a\textsubscript{i+1},...,a\textsubscript{j}.
We call this path a\textsubscript{i}, x\textsubscript{1}, x\textsubscript{2},..., x\textsubscript{d}, a\textsubscript{j}.
Length of this path must be greater than the length of former path, otherwise it will reduce the distance between 
s and t. Therefore we need at least n/2 vertices to involve all a\textsubscript{i} 's in the cycle, but we have less than
(n/2)-2 vertices at our disposal. Therefore, there will always be a vertex a\textsubscript{i} such that it is not included 
in any cycle formation.

\

\textbf{Algorithm}: Perform BFS on the given graph starting from s. There will be a layer in the BFS tree which will contain only 
one vertex.

\textbf{Explanation:} BFS Tree will have at least n/2 layers , because distance between s and t is atleast n/2. If 
we have atleast 2 vertices in each layer than it will result in nodes more than n.

\

This is the vertex we need to remove to disconnect s from t.

\textbf{Explanation:} There will be no edge that connects two components divided by this vertex because if there is an edge between 
two vertices then they are atmost 1 layer apart.

\

Because we are using BFS algorithm only, so time complexity of our algorithm is \emph{O(m+n)}
}
\end{homeworkProblem}

%----------------------------------------------------------------------------------------

\end{document}
